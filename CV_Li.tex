%%%%%%%%%%%%%%%%%%%%%%%%%%%%%%%%%%%%%%%%%
% Medium Length Professional CV
% LaTeX Template
% Version 2.0 (8/5/13)
%
% This template has been downloaded from:
% http://www.LaTeXTemplates.com
%
% Original author:
% Trey Hunner (http://www.treyhunner.com/)
%
% Important note:
% This template requires the resume.cls file to be in the same directory as the
% .tex file. The resume.cls file provides the resume style used for structuring the
% document.
%
%%%%%%%%%%%%%%%%%%%%%%%%%%%%%%%%%%%%%%%%%

%----------------------------------------------------------------------------------------
%	PACKAGES AND OTHER DOCUMENT CONFIGURATIONS
%----------------------------------------------------------------------------------------

\documentclass{resume} % Use the custom resume.cls style
\usepackage{fontspec, color, enumerate, sectsty}
\usepackage[normalem]{ulem}
\usepackage{hyperref}
\defaultfontfeatures{Mapping=tex-text}

\usepackage[left=0.75in,top=0.6in,right=0.75in,bottom=0.6in]{geometry} % Document margins
\newcommand{\tab}[1]{\hspace{.2667\textwidth}\rlap{#1}}
\newcommand{\itab}[1]{\hspace{0em}\rlap{#1}}
\name{Li Wang} % Your name

\address{Office: 353B Bessey Hall, Iowa State University, Ames, IA 50011 United States } % addess 
\address{Website: \url{https://lepisorus.github.io/}} % Your secondary addess (optional)
\address{(806) 319-0182  \\ lilepisorus@gmail.com} % Your phone number and email

\begin{document}

%----------------------------------------------------------------------------------------
%	EDUCATION SECTION
%----------------------------------------------------------------------------------------

\begin{rSection}{Education}

{\bf Institute of Botany, Chinese Academy of Sciences, Beijing, China} \hfill {\em 2011} 
\\ PhD Plant Evolution 
\\Thesis: Phylogeny, diversification and phylogeography of the derived fern genus \emph{Lepisorus}
%\\Advisor: Dr. Xianchun Zhang, Dr. Harald Schneider and Dr. Deyuan Hong\smallskip \\

{\bf Wuhan University, Wuhan, Hubei Province, China} \hfill {\em 2006} 
\\ BS Biotechnology 
%\\ Advisor: Dr. Shuangquan Huang




\end{rSection}
%----------------------------------------------------------------------------------------
%	GRANTS
%----------------------------------------------------------------------------------------

%\begin{rSection}{GRANTS}
%{\bf NSF National Plant Genome Initiative Postdoctoral Research Fellowship } \hfill {\em \$207,000}
%\\{Investigating Centromere Evolution in Diploid and Polyploid Gossypium} \hfill {\em 2013-2016} 

%\end{rSection}
%----------------------------------------------------------------------------------------
%	WORK EXPERIENCE SECTION
%----------------------------------------------------------------------------------------

\begin{rSection}{Professional Experience}

\begin{rSubsection}{Iowa State University}{April 2014 - present}{Postdoctoral Research Associate on Bioinformatics}{}
\item Principal Investigator:  Dr. Matt Hufford
\item Investigated demography and its effect on mutation load in maize and teosinte using high-depth whole genome re-sequencing data
\item Explored parallel adaptation to highland environments in maize populations
\item Conducting comparative genomic analyses of four de-novo assembled genomes of maize landraces
\item Mentored graduate students in molecular and bioinformatic techniques
\end{rSubsection}

\begin{rSubsection}{Iowa State University}{April 2014 - December 2015}{Postdoctoral Research Associate on Bioinformatics}{}
\item Principal Investigator:  Dr. Andrew Severin
\item Provided services at Genomic Informatics Center
\item Detected selection during soybean improvement
\end{rSubsection}

\begin{rSubsection}{Texas Tech University}{Feburary 2012 - March 2014}{Postdoctoral Research Associate on Bioinformatics}{}
%\item Principal Investigator:  Dr. Matt Olson
\item Studied gene expression and co-expression network differences resulting from latitude adaptation in balsam poplar with transcriptome data
\item Evaluated gene expression pattern and enriched pathways owing to sex dimorphism in dioecious poplars
\item Assessed DNA and protein evolutionary rates and detected signatures of selection
\item Mentored graduate students in molecular and bioinformatic techniques
\end{rSubsection}

\begin{rSubsection}{Institute of Botany, Chinese Academy of Sciences}{June 2007 - July 2011}{Research Assistant}{}
%\item Principal Investigator:  Dr. Xianchun Zhang 
\item Investigated phylogeny, phylogeography, polyploidization and reticulate evolution in ferns
\item Mentored multiple graduate students in experimental and analyses techniques
\end{rSubsection}

\begin{rSubsection}{Goettingen University, Germany}{February 2009 - November 2010}{Research Assistant}{}
%\item Principal Investigator: Dr. Harald Schneider 
\item Examined phylogeny, phylogeography, polyploidization and reticulate evolution in ferns
\end{rSubsection}

\begin{rSubsection}{Natural History Museum, London, UK}{December 2009, April 2010}{Research Assistant}{}
%\item Principal Investigator: Dr. Harald Schneider 
\item Inspected phylogeny, phylogeography, polyploidization and reticulate evolution in ferns
\end{rSubsection}

\begin{rSubsection}{Institute of Hydrobiology, Chinese Academy of Sciences}{June 2005 - December 2005}{Undergraduate Research}{}
\item Collected data of fish diversity to determine the effect of constructing the three gorges dam on Yangtze River 
%\item Research Advisor: D. Jianbo Chang 
\end{rSubsection}

\end{rSection}

%------------------------------------------------
\begin{rSection}{Skills} \itemsep -3pt
\begin{rSubsection}{Scripting languages}{}{}{}
\item Proficient in  Unix scripting/bash, python, perl, R, markdown and latex 
\end{rSubsection}

\begin{rSubsection}{Software proficiencies}{}{}{}
\item \textbf{Assembly}: Trinity, Falcon, MaSuRCA, Platanus, GenomeScope, Redundans  
\item \textbf{Annotation}: GeneMark, Augustus, Busco, CEGMA, Repeatmasker, BLAST, Trinotate, Blobtools
\item \textbf{Aligners}: Bowtie, BWA, Gmap, Gsnap, Hisat2
\item \textbf{Transcriptomics}: Trinity, Rsem, edgeR, WGCNA, mapMan
\item \textbf{Comparative genomics}: GATK, Bedtools, vcftools, PLINK, TASSEL, MEGA
\item \textbf{Machine learning}: scikit learn (regression, clustering algorithms)
\item \textbf{Population genomics}: PopGenome, ANGSD, NGSadmix, MSMC, PSMC, fastSimcoal, ms
\item \textbf{Phylogenetics}: Mafft, ClustalW, Muscle, RaxML, Phagon  
\item \textbf{Visualization}: Circos, Cytoscape, Photoshop
\item \textbf{Other}: Seqtk, Aspera, Bamtools, Bioawk, Cdbfasta, Fastqc, Gawk, Jellyfish, Picard Tools, Samtools, SRA-Toolkit, Tabix, Trimmomatic, Sickle, Vcftools, BLAT, Emboss etc 
\end{rSubsection}

\begin{rSubsection}{Lab proficiencies}{}{}{}
\item DNA and RNA extraction, bioNano, PCR, clone etc. 
\end{rSubsection}

\end{rSection}




\begin{rSection}{Coding Accomplishments}

\begin{rSubsection}{Publicly available pipelines and scripts }{}{}{}
\item {https://github.com/HuffordLab/Wang\_et\_al\.\_Demography -- computational pipeline for demography and its effect on mutation in maize}
\item {https://github.com/ISUgenomics/commonscripts -- Common genomic analyses}
\end{rSubsection}{}

\end{rSection}

%----------------------------------------------------------------------------------------
\begin{rSection}{Awards and Honors}

\begin{rSubsection}{2011}{}{}{}
\item{DEAN Award in Chinese Academy of Sciences, Beijing, China}
\end{rSubsection}

\begin{rSubsection}{2011}{}{}{}
\item{The excellent graduate student in Chinese Academy of Sciences, Beijing, China}
\end{rSubsection}

\begin{rSubsection}{2010-2011}{}{}{}
\item{BHPB-GUCAS Scholarship for excellent Ph.D. candidates, Beijing, China; \$2,000} 
\end{rSubsection}

\begin{rSubsection}{2010}{}{}{}
\item{The second prize for giving talks at students’ symposium in Natural History Museum, London, UK}
\end{rSubsection}

\begin{rSubsection}{2008 - 2010 }{}{}{}
\item{DAAD-CAS Joint Scholarship Program -- \textit{1,000 Euro per month for two years}, Goettingen, Germany}
\end{rSubsection}{}

\begin{rSubsection}{2009 - 2010}{}{}{}
\item{Merit Student by the Graduate University of Chinese Academy of Sciences, Beijing, China}
\end{rSubsection}

\end{rSection}
%----------------------------------------------------------------------------------------
\begin{rSection}{Conference Participation}{}{}
\begin{rSubsection}{Oral Presentation}{}{}{}
\item \textbf{Li Wang}, Harald Schneider, Xianchun Zhang. (July 2011). The rise of the Himalaya enforced the diversification of SE Asian ferns by altering the monsoon regimes. International Botany Conference, Melbourne, Australia
\item \textbf{Li Wang}, Harald Schneider, Qiaoping Xiang, Xianchun Zhang. (November 2010). Phylogeography of the alpine fern \emph{Lepisorus clathratus} on "the roof of the world". The Fifth Asian Fern Symposium, Shenzhen, China
\item \textbf{Li Wang}, Xianchun Zhang. (December 2008). Phylogeny of the paleotropical fern genus \emph{Lepisorus} (Polypodiaceae, Polypodiopsida) inferred from four chloroplast genome regions. Symposium on Academic study of Chinese ferns, Shenzhen, China

\end{rSubsection}

\begin{rSubsection}{Poster Presentation}{}{}{}
\item \textbf{Li Wang}, Emily Josephs, Lucas M. Roberts. Timothy Mathes Beissinger, Jeffrey Ross-Ibarra, Matthew Hufford. (March 2017). Detection of Convergent Highland Adaptation in maize landrace populations. 59th maize genetic conference, St. Luis, Missouri, USA
\item \textbf{Li Wang}, Timothy Mathes Beissinger, Jeffrey Ross-Ibarra, Matthew Hufford. (March 2016). Evolution of maize during post-domestication expansion across the Americas. 58th maize genetic conference, Jacksonville, Florida, USA
\item \textbf{Li Wang}, Timothy Mathes Beissinger, Jeffrey Ross-Ibarra, Matthew Hufford. (March 2015). Inference of Maize Population History during Migration to Highland Habitats. 57th maize genetic conference, St. Charles, Illinois, USA
\item \textbf{Li Wang}, Timothy Mathes Beissinger, Jeffrey Ross-Ibarra, Matthew Hufford. (January 2015). Inference of Maize Population History during Migration to Highland Habitats. Plant and Animal Genome Conference, San Diego, CA, USA
\item \textbf{Li Wang}, Matt Olson. (June 2013). Timing for success: exprssion phenotype and local adaptation related to latitude in the boreal forest tree. Evolution, Snowbird, Utah, USA
\end{rSubsection}

\begin{rSubsection}{Attended}{}{}{}
\item July 2009, Niche Evolution Conference, Zurich, Switzerland
\item August 2009, Systematics 2009, Leiden, Netherland 
\item May 2006, Symposium on Academic study of Chinese ferns, Shangri-La, China
\end{rSubsection}

\end{rSection}


\begin{rSection}{Leadership Skills}{}{}
\item Social activity manager of Postdoctor Association at Iowa State University
\item Participated in the Self-expression and Leadership Program at Landmark (\url{http://www.landmarkworldwide.com/})
\item Will take a leadership program for women scientists on a three-week expedition to Antarctica in February 2018
\end{rSection}

\begin{rSection}{Publications}{}{}
\begin{rSubsection}{\textbf{Published}}{}{}{}

\item \textbf{Li Wang}, Timothy Mathes Beissinger, Anne Lorant, Claudia Ross-Ibarra, Jeffrey Ross-Ibarra, Matthew Hufford. 2017. The interplay of demography and selection during maize domestication and expansion. \textbf{bioRxiv}. doi: https://doi.org/10.1101/114579. 

\item Timothy Beissinger, \textbf{Li Wang}, Kate Crosby, Arun Durvasula, Matthew B Hufford, Jeffrey Ross-Ibarra. 2016. Recent demography drives changes in linked selection across the maize genome. \textbf{Nature plants} 2: 16084. 

\item \textbf{Li Wang}, Peter Tiffin, Matthew Olson. 2014. Timing for success: exprssion phenotype and local adaptation related to latitude in the boreal forest tree, Populus balsamifera. \textbf{Tree Genetics and Genomes}. DOI: 10.1007/s11295-014-0731-3. 

\item Harald Schneider, Lijuan He, Jeannine Marquardt, \textbf{Li Wang}, Jochen Heinrichs, Sabine Hennequin, Xianchun Zhang. 2013. Exploring the origin of the latitudinal diversity gradient: contrasting the sister fern genera \textit{Phegopteris} and \textit{Pseudophegopteris}. \textbf{Journal of Systematics and Evolution} 51: 61-70.

\item Xianchun Zhang, Ran Wei, Hongmei Liu, Lijuan He, \textbf{Li Wang}, Gangming Zhang. 2013. Phylogeny and classification of the extant lycophytes and ferns from China. \textbf{Chinese Bulletin of Botany} 48: 119-137.


\item \textbf{Li Wang}, Harald Schneider, Xianchun Zhang, Qiaoping Xiang. 2012. The rise of the Himalaya enforced the diversification of SE Asian ferns by altering the monsoon regimes. \textbf{BMC Plant Biology} 12:210. DOI: 10.1186/1471-2229-12-210. 

\item \textbf{Li Wang}, Harald Schneider, Zhiqiang Wu, Lijuan He, Xianchun Zhang, Qiaoping Xiang. 2012. Indehiscent sporangia enable the accumulation of local fern diversity at the Qinghai-Tibetan Plateau. \textbf{BMC Evolutionary Biology} 12:158. DOI:10.1186/1471-2148-12-158.

\item \textbf{Li Wang}, Zhiqiang Wu, Nadia Bystriakova, Stephen W. Ansell, Qiaoping Xiang, Jochen Heinrichs, Harald Schneider, Xianchun Zhang. 2011. Phylogeography of the alpine fern \textit{Lepisorus clathratus} on “the roof of the world”. \textbf{PloS One} 6: e25896. DOI: 10.1371/journal.pone.0025896. 

\item \textbf{Li Wang}, Xinping Qi, Qiaoping Xiang, Jochen Heinrichs, Harald Schneider, Xianchun Zhang. 2010. Phylogeny of the paleotropical fern genus \textit{Lepisorus} (Polypodiaceae, Polypodiopsida) inferred from four chloroplast genome regions. \textbf{Molecular Phylogenetics and Evolution} 54: 211-225. 

\item \textbf{Li Wang}, Zhiqiang Wu, Qiaoping Xiang, Jochen Heinrichs, Harald Schneider, Xianchun Zhang. 2010. A molecular phylogeny and a revised classification of tribe Lepisoreae (Polypodiaceae) based on an analysis of four plastid DNA regions. \textbf{Botanical Journal of the Linnean Society} 162: 28-38. 

\item Hongmei Liu, \textbf{Li Wang}, Hui Zeng, Xianchun Zhang. 2008. Advances in the studies of lycophytes and monilophytes with reference to systematic arrangement of families distributed in China. \textbf{Journal of Systematics and Evolution} 46: 808-829. 

\item Xianchun Zhang, \textbf{Li Wang}, Xinping Qi. 2008. Additions to the pteridophyte flora of Xizang (IV). \textbf{Newsletter of Himalayan Botany} 41: 20-27. 
 
\item \textbf{Li Wang}, Xianchun Zhang. 2007. Pteridophyta. in Zhenyu Li, Lei Shi (editors), \textbf{Flora of Emei Mountain}. Beijing Science and Technology Press, Beijing, pp. 153-224. 

\item \textbf{Li Wang}, Xianchun Zhang. 2007. Chinese aquatic ferns. \textbf{China Nature} 138: 54-55. 


\end{rSubsection}
\begin{rSubsection}{\textbf{In Preparation}}{}{}{}

\item Purushottam R. Lomate, Veena Dewangan, Neha S. Mahajan, Yashwant Kumar, Abhijeet Kulkarni, \textbf{Li Wang}, Smita Saxena, Vidya S. Gupta, and Ashok P. Giri. 2017. Integrated transcriptomic and proteomic analyses revealed the participation of endogenous protease inhibitors in the regulation of protease gene expression in Helicoverpa armigera (\textbf{submitted})
\item Garrett M. Janzen, \textbf{Li Wang}, Matthew B. Hufford. Review: Adaptive Introgression Expanded the Genetic Base of Crops during post-Domestication Spread. (\textbf{Expected} \textbf{2017})
\item Brian Sanderson, \textbf{Li Wang}, Peter Tiffin, Zhi-qiang Wu, Matthew S. Olson. Males and females of \textit{Populus balsamifera} differ primarily in the expression of energy-related genes. (\textbf{Expected} \textbf{2017})



\end{rSubsection}
\end{rSection}


\begin{rSection}{Research Field Work}{}{}
\begin{rSubsection}{Nov 2015}{Mexican Central Plateau, Mexico}{}{}
\item Collected teosinte hybrid populations
\item Team work with collaborators from both University of California Davis and Mexican national laboratory of genomics for biodiversity 
\end{rSubsection}

\begin{rSubsection}{Jun-Jul 2012}{Alaska, US}{}{}
\item Collected leaves for RNA extraction in the common garden for balsam poplar at University of Alaska, Fairbanks
\item Cooperated with researchers in University of Alaska, Fairbanks
\end{rSubsection}

\begin{rSubsection}{Jul 2010}{Altai Mountains, Russia}{}{}
\item Studied fern biodiversity in the region of Altai Mountains
\item Worked jointly with Russian collaborators
\end{rSubsection}

\begin{rSubsection}{Jun 2010}{Alps, Switzerland and France}{}{}
\item Explored fern biodiversity in the Alps and collected populations of several fern species
\item Worked closely with collaborators from Natural History Museum, UK
\end{rSubsection}

\begin{rSubsection}{Nov 2008}{Tropical and subtropical regions, China}{}{}
\item Examined fern biodiversity in Hainan and Guangdong Province
\item Worked together with researchers from Zhongshan University, China
\end{rSubsection}

\begin{rSubsection}{May 2008}{Central Inner Mongolia, China}{}{}
\item Collected populations of \emph{Lepisorus clathratus} in the northern desert region of China 
\end{rSubsection}

\begin{rSubsection}{Dec 2007}{Southeast of Yunnan, China}{}{}
\item Investigated plant biodiversity at border of China and Vietnam
\item Discovered several new species
\item Collaborated with researches from Kunming Botanical Garden, China
\end{rSubsection}

\begin{rSubsection}{Oct 2007}{The Himalayas and Tibetan Plateau, China}{}{}
\item Investigated fern biodiversity at southeast of Tibet as a member of expedition team organized by World Wildlife Foundation
\item Collaborated with many researches from different discipline 
\end{rSubsection}

\begin{rSubsection}{Aug 2007}{Hengduan Mountains, China}{}{}
\item Explored fern biodiversity 
\end{rSubsection}

\begin{rSubsection}{May 2005}{The three Gorges region, China}{}{}
\item Studied fish biodiversity 
\item Cooperated with local fishermen and administrative officers
\end{rSubsection}

\end{rSection}



\begin{rSection}{References}
\itab{\textbf{Dr. Matt Hufford}} \tab{} \tab{\textbf{Dr. Jeffrey Ross-Ibarra}}
\\ \itab{Assisstant Professor of Biological Sciences} \tab{}  \tab{Professor of Biological Sciences}
\\ \itab{Ecology, Evolution, \& Organismal Biology} \tab{}  \tab{Department of Plant Sciences}
\\ \itab{Iowa State University} \tab{}  \tab{University of California, Davis}
\\ \itab{Ames, Iowa 50011} \tab{}  \tab{Davis, CA 95616}
\\ \itab{Tel: (515) 294-8511} \tab{}  \tab{Tel: (530) 752-1152}
\\ \itab{Email: mhufford@iastate.edu } \tab{}  \tab{Email: rossibarra@ucdavis.edu} 
\\
\\
\itab{\textbf{Dr. Andrew Severin}} \tab{} \tab{\textbf{Dr. Matt Olson}}
\\ \itab{Scientist I/Facility Manager} \tab{} \tab{Associate Professor}
\\ \itab{Genome Informatics Facility} \tab{} \tab{Department of Biological Sciences}
\\ \itab{Iowa State University} \tab{} \tab{Texas Tech University}
\\ \itab{Ames, Iowa 50011} \tab{} \tab{Lubbock, TX 79409}
\\ \itab{Tel: (515) 294-1320} \tab{} \tab{Tel: (806) 834-7252}
\\ \itab{Email: severin@iastate.edu} \tab{} \tab{Email: matt.olson@ttu.edu}


\textbf{}

\end{rSection}

\end{document}
